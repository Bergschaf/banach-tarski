\chapter{Applications of Jensen's inequality}

In this chapter, $h$ denotes the function $h(x) := x \log \frac{1}{x}$ for $x \in [0,1]$.

\begin{lemma}[Concavity]\label{concave}
  \lean{Real.strictConcaveOn_negMulLog}\leanok
  $h$ is strictly concave on $[0,1]$.
\end{lemma}

\begin{proof} \leanok Check that $h'$ is strictly monotone decreasing.
\end{proof}

\begin{lemma}[Jensen]\label{jensen}
  \lean{Real.sum_negMulLog_le} \leanok
  If $S$ is a finite set, and $\sum_{s \in S} w_s = 1$ for some non-negative $w_s$, and $p_s \in [0,1]$ for all $s \in S$, then
  $$ \sum_{s \in S} w_s h(p_s) \leq h(\sum_{s \in S} w_s p_s).$$
\end{lemma}

\begin{proof} \uses{concave}\leanok Apply Jensen and Lemma \ref{concave}.
\end{proof}

\begin{lemma}[Converse Jensen]\label{converse-jensen}
  \lean{Real.sum_negMulLog_eq_iff}\leanok
If equality holds in the above lemma, then $p_s = \sum_{s \in S} w_s h(p_s)$ whenever $w_s \neq 0$.
\end{lemma}

\begin{proof} \uses{concave}\leanok Need some converse form of Jensen, not sure if it is already in Mathlib.  May also wish to state it as an if and only if.
\end{proof}

\begin{lemma}[moin]\label{unnoetig}
\end{lemma}
\begin{proof}\uses{concave}
Tja.
\end{proof}
\section{Wahl der Rotationen und Zerlegung des Großteils der Kugel}
Zunächst definieren wir, was wir unter einer Rotation formal verstehen.
\begin{definition}[Rotationsmatrix]\label{def:rot}
Eine Rotation der Sphäre wird durch Rotationsmatrizen blablabla gegeben
\end{definition}


Wir wählen nun konkret zwei Rotationen der Kugel, mit welchen wir die Zerlegung im Satz konstruieren werden.
\begin{definition}\label{def:konk_rot}
\lean{blublu}\leanok
Die konkreten Rotationsmatrizen sind... 
\end{definition}
\begin{lemma}\label{lem:konk_rot_sind_rot}
Dies sind in der Tat Rotationen
\end{lemma}
\begin{proof}\uses{def:rot,def:konk_rot}
Direkte Rechnung.
\end{proof}

Diese erzeugen uns somit eine Untergruppe der Gruppe der invertierbaren $3\times3$-Matrizen.
\begin{lemma}\label{lem:konk_rot_erzeugt}
Die beiden
\end{lemma}
\begin{proof}\uses{def:konk_rot,lem:konk_rot_sind_rot}

\end{proof}

Nun hat jedes Element dieser erzeugten Untergruppe eine besondere Darstellung.
\begin{lemma}\label{lem:darst_von_rot}
\end{lemma}
\begin{proof}\uses{lem:konk_rot_erzeugt,def:konk_rot}

\end{proof}
Damit können wir zeigen, dass diese Untergruppe an Rotationen eine freie Gruppe in zwei Erzeugern ist.

\begin{definition}\label{def:freie_grp}
Eine Freie Gruppe in Erzeugern $M$ ist...
\end{definition}

\begin{theorem}\label{thm:freie_grp_an_rot}
Die von unseren konkreten Rotationen aus \ref{def:konk_rot} erzeugte untergruppe ist eine Freie Gruppe
\end{theorem}
\begin{proof}\uses{lem:darst_von_rot,def:freie_grp}
chillig.
\end{proof}
\section{Verdoppeln der freien Gruppe}
\section{Die restlichen Punkte einfangen}
\begin{theorem}[Banach-Tarski]\label{thm:BanachTarski}
Man kann die Kugel nur durch Zerschneiden und drehen und zusammensetzen verdoppeln.
\end{theorem}
\begin{proof}
TBD
\end{proof}
