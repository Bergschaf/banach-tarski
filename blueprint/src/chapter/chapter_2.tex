\chapter{Freie Gruppe der Rotationen}


\section{Wahl der Rotationen und Zerlegung des Großteils der Kugel}


%\begin{definition}[Rotationsmatrix]\label{def:rot}
%Eine Rotation der Sphäre wird durch Rotationsmatrizen blablabla gegeben
%\end{definition}


Wir wählen konkret zwei Rotationen der Kugel, mit welchen wir die Zerlegung im Satz konstruieren werden.
\begin{definition}\label{def:konk_rot}
\leanok \lean{matrix_a}
Unsere Rotationsmatrizen sind: 

\end{definition}

\begin{lemma}[Invertierbarkeit von A und B] \label{lemma:a_b_invertierbar}
Es gilt $det A\neq 0$ und $det B\neq 0$ und damit sind $A$ und $B$ invertierbar.
\lean{matrix_a_det_neq_zero}\leanok
\end{lemma}
\begin{proof} \uses{def:konk_rot} \leanok 
Folgt durch Nachrrechnen.
\end{proof}

$A$ und $B$ erzeugen uns daher eine Untergruppe der Gruppe der invertierbaren $3\times3$-Matrizen.
\begin{definition}\label{def:konk_rot_erzeugt} \uses{lemma:a_b_invertierbar}
$G$ bezeichnet die von $A$ und $B$ erzeugte Untergruppe.
\leanok \lean{G}
\end{definition}

\begin{lemma}\label{lem:darst_von_rot_matrix}
Die Elemente von $G$ können wir durch eine bestimmte Schreibweise darstellen.
\lean{general_word_form_exists} \leanok
Matrizen einfügen
\end{lemma}
\begin{proof}\uses{def:konk_rot_erzeugt,def:konk_rot}
sorry
\end{proof}

\begin{lemma}[Konkrete darstellung der Drehungen]\label{lem:darst_von_rot_res}
\lean{lemma_3_1}\leanok
Wenn $\rho : \mathbb{R}³\Rightarrow\mathbb{R}³$ ein Ausdruck in $G$ der Länge 
$n$ in reduzierter Form ist, dann ist $\rho(0,1,0)$ von der folgenden Form, wobei
$a, b$ und $c$ ganze Zahlen sind: $\rho(0,1,0)=\frac{1}{3^n}(a\sqrt{2},b,c\sqrt{2})$.
\end{lemma}
\begin{proof} \uses {lem:darst_von_rot_matrix}
\leanok Diese Behauptung folgt aus den Erzeugermatrizen
und durch konkretes Multiplizieren eines reduzierten Wortes an (0,1,0).
\end{proof}
Damit können wir zeigen, dass diese Untergruppe an Rotationen eine freie Gruppe in zwei Erzeugern ist.


\begin{definition}\label{def:freie_grp}
Eine freie Gruppe $G$ ist eine Gruppe, in welcher zwei Wörter auf einer spezifischen Erzeugermenge unterschiedlich 
sind, außer ihre Gleichheit folgt aus den Gruppenaxiomen.
\leanok
\end{definition}

\begin{theorem}\label{thm:freie_grp_an_rot}
Die von unseren konkreten Rotationen aus \ref{def:konk_rot} erzeugte Untergruppe $G$ ist eine freie Gruppe.
\lean{freeGroup} \leanok
\end{theorem}
\begin{proof}\uses{lem:darst_von_rot_res,def:freie_grp}
chillig.
\end{proof}


