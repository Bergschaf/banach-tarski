\chapter{Freie Gruppe der Rotationen}

\begin{lemma}[Konkrete darstellung der Drehungen]\label{Konkrete_Darstellung_Drehungen}
\lean{banach_tarski.Lemma_3_1.lemma_3_1}\leanok
Wenn $\rho : \mathbb{R}³\Rightarrow\mathbb{R}³$ ein Ausdruck in $G$ der Länge 
$n$ in reduzierter Form ist, dann ist $\rho(0,1,0)$ von der folgenden Form, wobei
$a, b$ und $c$ ganze Zahlen sind: $\rho(0,1,0)=\frac{1}{3^n}(a\sqrt{2},b,c\sqrt{2})$.
\end{lemma}

\begin{proof}[halloo]\leanok Diese Behauptung folgt aus den Erzeugermatrizen
     und durch konkretes Multiplizieren eines reduzierten Wortes an (0,1,0).
\end{proof}