\chapter{Freie Gruppe der Rotationen}


\section{Wahl der Rotationen und Zerlegung des Großteils der Kugel}

Zunächst definieren wir, was wir unter einer Rotation formal verstehen.
\begin{definition}[Rotationsmatrix]\label{def:rot}
Eine Rotation der Sphäre wird durch Rotationsmatrizen blablabla gegeben
\end{definition}


Wir wählen nun konkret zwei Rotationen der Kugel, mit welchen wir die Zerlegung im Satz konstruieren werden.
\begin{definition}\label{def:konk_rot}
Unsere Rotationsmatrizen sind: 
$A=\frac{1}{3}\begin{pmatrix} 3 & 0 & 0 //
                              0 & 1 & 2\sqrt{2} //

\end{definition}

\begin{lemma}\label{lem:konk_rot_sind_rot}
Dies sind in der Tat Rotationen
\end{lemma}
\begin{proof}\uses{def:rot,def:konk_rot}
Direkte Rechnung.
\end{proof}

Diese erzeugen uns somit eine Untergruppe der Gruppe der invertierbaren $3\times3$-Matrizen.
\begin{definition}\label{def:konk_rot_erzeugt}
Die beiden
\end{definition}


Nun hat jedes Element dieser erzeugten Untergruppe eine besondere Darstellung.
\begin{lemma}\label{lem:darst_von_rot_matrix}
\end{lemma}
\begin{proof}\uses{def:konk_rot_erzeugt,def:konk_rot}
sorry
\end{proof}

\begin{lemma}[Konkrete darstellung der Drehungen]\label{lem:darst_von_rot_res}
\lean{banach_tarski.Lemma_3_1.lemma_3_1}\leanok
Wenn $\rho : \mathbb{R}³\Rightarrow\mathbb{R}³$ ein Ausdruck in $G$ der Länge 
$n$ in reduzierter Form ist, dann ist $\rho(0,1,0)$ von der folgenden Form, wobei
$a, b$ und $c$ ganze Zahlen sind: $\rho(0,1,0)=\frac{1}{3^n}(a\sqrt{2},b,c\sqrt{2})$.
\end{lemma}
\begin{proof} \uses {lem:darst_von_rot_matrix}
\leanok Diese Behauptung folgt aus den Erzeugermatrizen
und durch konkretes Multiplizieren eines reduzierten Wortes an (0,1,0).
\end{proof}
Damit können wir zeigen, dass diese Untergruppe an Rotationen eine freie Gruppe in zwei Erzeugern ist.


\begin{definition}\label{def:freie_grp}
Eine Freie Gruppe in Erzeugern $M$ ist...
\end{definition}

\begin{theorem}\label{thm:freie_grp_an_rot}
Die von unseren konkreten Rotationen aus \ref{def:konk_rot} erzeugte untergruppe ist eine Freie Gruppe
\end{theorem}
\begin{proof}\uses{lem:darst_von_rot_res,def:freie_grp}
chillig.
\end{proof}


