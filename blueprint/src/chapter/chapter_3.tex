\chapter{Verdoppeln einer Kugel}

\section{Kugel}
\begin{definition}[Einheitskugel ohne Mittelpunkt]\label{def:kugel_ohne_mittelpunkt}
Sei $L=\{(x,y,z):x²+y²+z²\leq1\}$ die Einheitskugel. Wir definieren $L'=L\backslash {(0,0,0)}$ als Einheitskugel ohne Mittelpunkt.
\lean{L}
\leanok
\end{definition}

\begin{definition}[Orbit] \label{def:orbit} \uses{def:konk_rot_erzeugt}
Zwei Punkte $a$ und $b$ gehören zum selben Orbit, genau dann, wenn ein $\rho$ in $G$ existiert, sodass $\rho(a)=b$ gilt.
\leanok \lean{same_orbit}
\end{definition}

\begin{lemma}[Abzählbarkeit aller Orbits] \label{lemma:all_orbits_countable} 
Die Menge aller Orbits ist abzählbar.
\leanok \lean{all_orbits_countable}   
\end{lemma}
\begin{proof}
    \uses{def:orbit}
\end{proof}

%\begin{definition}[Auswahlfunktion] \label{def:auswahlfunktion}
%Sei $A$ eine Menge nichtleerer Mengen. Dann heißt $F:A\rightarrow\bigcup\limits_{I\in A}I$ eine Auswahlfunktion von $A$,
%falls zu jedem Element $I\in A$ ein Element $F(I)\in I$ zugeordnet wird.    
%\end{definition}  

%\begin{theorem}[Auswahlaxiom] \label{theorem:auswahlaxiom}
%Zu jeder Menge $A$ bestehend aus nichtleeren Mengen gibt es eine Auswahlfunktion.
%\end{theorem}
%\begin{proof} \uses{def:auswahlfunktion}
%Folgt aus der Definition der Auswahlfunktion.
%\end{proof}    

\begin{lemma}[repräsentative Punkte] \label{theorem:rep_punkte}
Wir können uns aus jedem Orbit einen repräsentativen Punkt auswählen.
\lean{rep_punkte} \leanok
\end{lemma}
\begin{proof} \uses{def:orbit} \leanok
Dies folgt direkt mit dem Auswahlaxiom.
\end{proof}

\begin{definition}[Menge aller  Repräsentanten] \label{def:menge_rep_punkte} \uses{theorem:rep_punkte}
$M$ ist die Menge aller ausgwählten repräsentativen Punkte.
\lean{M} \leanok
\end{definition} 

Wir können nun jeden Punkt aus $L'$ erreichen, indem wir eine Rotation aus $G$ auf ein bestimmtes Element in $M$ anwenden.

\section{Ein Teil der Einheitskugel duplizieren}

Wir wollen aber, dass jeder Punkt aus $L'$ nur von einer Rotation in $G$ erreicht wird. Daher zerlegen wir nun $L'$ entsprechend
der Rotationen, welche einen Punkt treffen. Ein Punkt, der von mehreren Rotationen getroffen wird, kann dabei in mehreren Mengen landen.

%\begin{lemma}[Punkte auf Rotationsachse] \label{lemma:punkte_auf_rot_achse}
%Alle Punkte auf einer Rotationsachse können durch mehrere Rotationen getroffen werden.
%\end{lemma}
%\begin{proof}
%Steht noch aus.
%\end{proof}

\begin{definition}[Fixpunkte] \label{def:fixpunkte}
Sei $Y$ eine Menge und $f:Y\rightarrow Y$ eine Funktion. Dann heißt ein Punkt $y\in Y$ Fixpunkt, falls er die Gleichung $f(y)=y$
erfüllt.
\leanok \lean{fixpunkte}
\end{definition}

\begin{definition}[Menge aller Fixpunkte] \label{def:menge_fixpunkte} \uses{def:fixpunkte,def:kugel_ohne_mittelpunkt}
Bezeichne mit $D$ die Menge aller Punkte in $L'$, welche Fixpunkte der Rotationen in $G$ sind.
\leanok \lean{D}
\end{definition}

\begin{lemma}[Abzählbarkeit von G] \label{lemma:G_abzaehlbar}
$G$ ist abzählbar.
\lean{g_countable} \leanok
\end{lemma}
\begin{proof} \uses{def:konk_rot_erzeugt, thm:freie_grp_an_rot}
Steht noch aus.
\end{proof}

\begin{definition}[Genau eine Rotationsachse] \label{lemma:eine_rot_achse}
Jede Rotation in $G$ hat genau eine Rotationsachse.
\lean{rotationsAchse} \leanok \uses{def:konk_rot_erzeugt,def:konk_rot}
\end{definition}


\begin{lemma}[Abzählbarkeit Rotationsachsen] \label{lemma:abz_rot_achsen}
Die Rotationsachsen liegen auf abzählbar vielen Linien.
\end{lemma}
\begin{proof} \uses{def:menge_fixpunkte,lemma:G_abzaehlbar,lemma:eine_rot_achse, lemma:abz_menge_rep_punkte}
Steht noch aus.
\end{proof}

%\begin{lemma}[Lebesgue Maß D] \label{lemma:lebesgue_mass_D}
%$D$ hat ein Lebesgue-Maß von $0$.
%\end{lemma}
%\begin{proof} \uses{lemma:abz_fixpunkte}
%Steht noch aus.
%\end{proof}

Daher kann fast jeder Punkt in $L'$ durch eine bestimmte Rotation erreicht werden. Wir betrachten nun zunächst eine 
Zerlegung von $L'\backslash D$ und kümmern uns später um die Fixpunkte.

\begin{definition}[Vereinigung X] \label{def:vereinigung_x} \uses{def:menge_rep_punkte,def:konk_rot}
$X=\bigcup\limits_{k=1}^{\infty}A^{-k}M$. $X$ ist also die Menge aller Elemente von $M$, welche ausschließlich aus Rotationen
um $A^{-1}$ bestehen.
\leanok \lean{X}
\end{definition}

\begin{definition}[Zerlegung in Mengen] \label{def:zerlegung_L_D} \uses{def:menge_rep_punkte,def:vereinigung_x,def:konk_rot}
$P_1=S(A)M\cup M\cup X$ \\
$P_2=S(A^{-1})M\backslash X$ \\
$P_3=S(B)M$ \\
$P_4=S(B^{-1})M$ \\
\leanok
\lean{P₁}
\end{definition}

\begin{lemma}[Vereinigung der Zerlegung] \label{lemma:vereinigung_zerlegung}
$L'\backslash D=P_1\cup P_2\cup P_3\cup P_4$
\leanok \lean{union_parts}
\end{lemma}
\begin{proof} \uses{def:zerlegung_L_D,def:menge_fixpunkte,def:kugel_ohne_mittelpunkt}
Steht noch aus.
\end{proof}

\begin{lemma}[Drehung zerlegte Mengen] \label{lemma:rot_zerlegte_mengen}
$AP_2=P_2\cup P_3\cup P_4$
$BP_4=P_1\cup P_2\cup P_4$
\leanok \lean{rot_A_P₂}
\end{lemma}
\begin{proof} \uses{def:zerlegung_L_D,def:konk_rot}
Steht noch aus.
\end{proof}

\begin{lemma}[Verdopplung L' \\D] \label{lemma:verdopplung_L_D}
$L'\backslash D=P_1\cup AP_2$
$L'\backslash D=P_3\cup BP_4$
\end{lemma}
\begin{proof} \uses{lemma:rot_zerlegte_mengen,lemma:vereinigung_zerlegung}
Steht noch aus.
\end{proof}

\begin{lemma}[Abzählbarkeit der repräsentativen Punkte] \label{lemma:abz_menge_rep_punkte}
Die Menge der Repräsentativen Punkte ist abzählbar.
\uses{def:menge_rep_punkte,lemma:all_orbits_countable}
\leanok \lean{M_countable}
\end{lemma}
\begin{proof}
    TODO
\end{proof}


Nun haben wir eine Zerlegung, welche uns erlaubt, die Kugel bis auf ihren Mittelpunkt und den Punkte auf den Rotationsachsen zu duplizieren.

\section{Fixpunkte und der Mittelpunkt}

\begin{definition}[Äquidekomponierbar] \label{def:aequidekomponierbar} 
Zwei Mengen $C$ und $D$ heißen äquidekomponierbar, wenn $C$ in endlich viele Teile zerlegt werden kann, welche durch
Rotationen und Translationen zu $D$ wieder zusammengefügt werden können.
\leanok \lean{equidecomposable}
\end{definition}

\begin{lemma}[Äquidekomponierbarkeit von L'\\D und L'] \label{lem:aequidekomponierbarkeit}
$L'\backslash D$ und $L'$ sind Äquidekomponierbar
\end{lemma}
\begin{proof}\uses{def:aequidekomponierbar,lemma:abz_rot_achsen}
Da die Punkte in $D$ auf abzählbar vielen Achsen liegen, kann man eine Linie $l$ durch den Ursprung finden, die $D$ nicht durchläuft.
Außerdem gibt es einen Winkel $\theta$, sodass eine Drehung um $\theta$ um $l$ keinen Punkt in $D$ auf einen anderen Punkt in $D$ abbildet.
Definiere $E=D\cup \rho(D)\cup \rho²(D)\cup \rho³(D)\cup...$. Es folgt sofort, dass
$L'=E\cup (L' \backslash E)$, was äquidekomponierbar mit $\rho(E) \cup (L' \backslash E)$ ist.
Aus der Definition von $E$ folgt, dass $\rho(E)=E\backslash D$ also gilt $\rho(E)\cup (L'\backslash E)=(E\backslash D)\cup (L'\backslash E)=
L'\backslash D$. Daher folgt, dass $L'$ mit $L'\backslash D$ äquidekomponierbar ist.
\end{proof}

Da wir nun wissen, dass $L'\backslash D$ mit $L'$ äquidekomponierbar ist, sind die Fixpunkte der Rotationen in $G$ kein Problem mehr.
Daher kümmern wir uns nun um den Mittelpunkt.

\begin{lemma}[Pi und Wurzel 2 haben kein gemeinsames vielfaches] \label{lemma:ncm_pi_sqrt_2} % no common multiple
$\nexists p,q\in \mathbb{Z} \sqrt{2}\cdot \frac{p}{q}=\pi$
\lean{pi_sqrt_two}
\leanok
\end{lemma}
\begin{proof}
Angenommen es gäbe $p,q\in \mathbb{Z}$ mit $\sqrt{2}\cdot \frac{p}{q}=\pi$. Nach Definition von $\cos^{-1}$ auf dem Intervall $[0;2\pi)$ gilt $\cos^{-1}(-1)=\pi$ und damit wäre $\cos^{-1}(-1)=
\sqrt{2}\cdot \frac{p}{q}$. Dies ist äquivalent zu $-1=\cos(\sqrt{2}\cdot \frac{p}{q})$. Mit der Taylorreihe des cosinus würde 
$\sum_{n=0}^{\infty}{\frac{(-1)^n}{(2n)!}(2\cdot \frac{p^2}{q^2})^n}=-1$ gelten. Es gilt:\\
$\sum_{n=0}^{\infty}{\frac{(-1)^n}{(2n)!}(2\cdot \frac{p^2}{q^2})^n}=-1$
$\Leftrightarrow \sum_{n=0}^{\infty}\frac{(2\cdot \frac{p^2}{q^2})^{2n}}{(2n)!}-\sum_{n=0}^{\infty}\frac{(2\cdot \frac{p^2}{q^2})^{2n+1}}{(2n)!}$
\end{proof}

\begin{lemma}[Äquidekomponierbarkeit Kreis] \label{lemma:aequi_kreis}
Ein Kreis ist äquidekomponierbar mit einem Kreis ohne einen bestimmten Punkt.
\lean{equi_kreis} \leanok
\end{lemma}
\begin{proof} \uses{def:aequidekomponierbar,lemma:ncm_pi_sqrt_2}
\leanok
Wir kümmern uns um den Einheitskreis $S¹=\{(x,y):x²+y²=1\}$ ohne ${(1,0)}$. Wir verwenden den Einheitskreis, um die Notation einfacher zu halten; 
das folgende Argument kann aber auf jeden beliebigen Kreis übertragen werden.
Sei weiter $A=\{(cos n,sin n):n\in\mathbb{Z}\}$. Da $\pi$ irrational ist, gilt $(cos n,sin n)\neq (cos m, sin m)$ für alle $n,m\in\mathbb{Z}$
mit $n\neq m$. Daher ist $A$ abzählbar unendlich und enthält $(1,0)$ nicht. Sei $B=(S¹\backslash\{(1,0)\})\backslash A$.
Rotiere nun $A$ um den Ursprung um $-1$ Einheiten. Bezeichne diese gedrehte Menge mit $A'$. Diese Rotation bildet $(cos 1,sin 1)$ auf $(1,0)$
ab, was gerade unserem fehlenden Punkt entspricht. Da jeder Punkt in $A$ eine Einheit neben den Punkten aus $A'$ liegt und $A$ abzählbar unendlich
ist, liegt jeder Punkt, der ursprünglich in $A$ war auch in $A'$. Daraus folgt $A'=A\cup \{(1,0)\}$ und daher ist
$S¹\backslash \{(1,0)\}=A\cup B$ äquidekomponierbar mit $A'\cup B=S¹$.
\end{proof}


\begin{lemma}[Äquidekomponierbarkeit Subset] 
\label{lemma:equi_subset}
\leanok
\lean{equidecomposable_subset}
\end{lemma}
\begin{proof}
\uses{def:aequidekomponierbar}
\end{proof}

\begin{theorem}[Äquidekomponierbarkeit Kugel] \label{theorem:aequi_kugel}
Eine Kugel ohne ihren Mittelpunkt ist äquidekomponierbar mit der vollständigen Kugel.
\lean{equi_kugel} \leanok
\end{theorem}
\begin{proof} \uses{lemma:aequi_kreis, lemma:equi_subset}
Die Konstruktion eines Kreises im Inneren der Kugel, welcher den Mittelpunkt der Kugel beinhaltet, liefert zusammen mit Lemma \ref{lemma:äqui_kreis},
dass die Kugel ohne ihren Mittelpunkt äquidekomponierbar mit der vollständigen Kugel ist.
\end{proof}

\section{Der endgültige Beweis}

Nachdem wir die notwendigen Details zusammen haben, können wir nun das gesamte Paradoxon zeigen.
\begin{theorem}[Banach-Tarski]\label{thm:BanachTarski}
Eine Kugel ist äquidekomponierbar mit zwei Kopien ihrer selbst.
\end{theorem}
\begin{proof} \uses{lem:aequidekomponierbarkeit,theorem:aequi_kugel,lemma:verdopplung_L_D}
Im zweiten Unterkapitel haben wir gezeigt, dass die Kugel ohne ihren Mittelpunkt und den Punkten auf den Rotationsachsen äquidekomponierbar
mit zwei Kopien ihrer selbst ist. Mit Lemma \ref{lem:äquidekomponierbarkeit} folgt, dass die Kugel ohne ihren Mittelpunkt äquidekomponierbar
mit zwei Kopien von sich ist. Nach Theorem \ref{lemma:äqui_kugel} folgt nun, dass die vollständige Kugel mit zwei Kopien von sich äquidekomponierbar
ist.
\end{proof}